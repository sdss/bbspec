%- Strawman BigBOSS spectrograph requirements based upon
%- extraction pipeline requirements

\title{BigBOSS Spectrograph Requirements from Extraction Pipeline Requirements}
\author{Stephen Bailey, Adam Bolton, Ted Kisner, David Schlegel}
\date{\today}

\documentclass[12pt]{article}

\begin{document}
\maketitle

This note documents the BigBOSS spectrograph requirements that derive
from extraction pipeline requirements, which in turn derive from
science requirements.  The initial version of this note is primarily
to define what parameters are important and what needs to be quantified
for the real requirements document.

%- PSF
\section{PSF}

In the following, the point spread function (PSF) $P$ is the projection
onto the CCD of a delta function in wavelength for a single fiber.
It is normalized $\int P(\vec x) d \vec x = 1$
and it is assumed that $P(\vec x) \ge 0$ for all $\vec x$.
{\it i.e.,} amplifier cross talk, which can cause an effective negative PSF,
is handled as a separate requirement (\S \ref{sec:ccd_amp_xtalk})
which is stringent enough (TBC) to make it not matter for the other metrics.

%- PSF Stability
\subsection{PSF Stability}

The stability of the PSF from time $t$ to $t^\prime$ may be characterized by:
\begin{equation}
    \Delta P(t-t^\prime) =
        { \int P(\vec x, t) P(\vec x, t^\prime) d \vec x \over
          \int P(\vec x, t) P(\vec x, t) d \vec x }
\end{equation}
$\Delta P = 1$ for a perfectly stable PSF, and is $< 1$ for PSF changes
which matter to the extraction pipeline.

{\bf Requirement}: The PSF shall be stable to $\Delta P(t-t^\prime) > 0.99$
for time intervals between calibrations, for all telescope and actuator
pointings.

Comments:
\begin{itemize}
    \item it is possible for the PSF to change in a way which preserves
        $\Delta P = 1$.  We should think carefully about this case and
        whether it matters.  e.g. $P = 1, 1, 1$ vs. $P^\prime = 1, 0.9, 1.1$
    \item Should we also specify an absolute time scale for the stability,
        e.g. $\Delta P > 0.99$ over the course of a night?
    \item Since the PSF has both a location and a shape, this requirement is
        also a requirement on stability of the trace positions and wavelength
        solutions.  We could separately specify that for convenience, but
        I don't think that is technically necessary.
\end{itemize}


%- PSF Overlap
\subsection{PSF Overlap}

The PSF overlap between two fibers $i$ and $j$ at wavelengths
$\lambda, \lambda^\prime$ may be characterized by:
\begin{equation}
    P_\times =
        { \int P_i(\vec x, \lambda) P_j(\vec x, \lambda^\prime) d \vec x \over
          \int P_i(\vec x, \lambda) P_i(\vec x, \lambda) d \vec x }
\end{equation}

{\bf Requirement}: $P_\times < 0.01$ for all fibers $i, j$ and
wavelengths $\lambda, \lambda^\prime$.

Comments:
\begin{itemize}
    \item Is this what we mean by ``less than 1\% PSF overlap?''  Or do
        the squared PSFs mean that we want $P_\times < 0.01^2$ ?
    \item It is expected that the dominant overlap would be between
        adjacent fibers and $\lambda \approx \lambda^\prime$.
\end{itemize}

%- PSF Size
\subsection{PSF Size}

The size of the PSF on the CCD can be characterized by the radius of the
encircled energy, e.g. EE99 = 99\% of the energy is within this radius.

{\bf Requirement}: EE99 $< 20$ pixels.

Comments:
\begin{itemize}
    \item Larger PSFs mean less dense projection matrices $A$ and thus
        more computing time.  {\it i.e.,} even with infinite CCD real estate and no
        PSF overlap, we would still want a requirement on the PSF size.
    \item Should we specify E50, E90, E95, E99, and E99.9?
\end{itemize}

%- Fiber geometry
\section{Fiber Geometry}

{\bf Requirement}: There shall be gaps between the fibers as projected
onto the CCD at least every 30 fibers such that the PSFs from fibers
across the gaps shall have $P_\times < 0.001$.

Comments:
\begin{itemize}
    \item There are two parameters to quantify here: the maximum number
        of fibers in a bundle, and the allowable crosstalk across bundles.
    \item This enables the extraction problem to be factored into bundles
        of fibers instead of having to simultaneously solve the entire CCD,
        significantly reducing the computational costs.
        $m \times O(n^3)$ vs. $O((m \times n)^3)$
\end{itemize}

%- CCD Properties
\section{CCD properties}
\subsection{Read Noise}

\subsection{Fraction of Bad Pixels / Columns}

\subsection{Dark and Bias Properties}
Linearity with time, ability of overscan to predict offset level of
active area, ...

\subsection{Amplifier Cross Talk}
\label{sec:ccd_amp_xtalk}
{\bf Requirement}: The cross talk between amplifiers shall be less than 0.1\%.

\section{Also see...}

\begin{itemize}
    \item BigBOSS pipeline design document \\
        (email from David Schlegel, July 20, 2011)
    \item Draft of BigBOSS requirements: \\
        https://bigboss.lbl.gov/trac/wiki/BBReqs/Telecon-21-7-2011
\end{itemize}


\end{document}
